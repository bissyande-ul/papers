\documentclass[conference]{IEEEtran}
\IEEEoverridecommandlockouts

\usepackage{graphicx}
\usepackage{epstopdf}

\usepackage{amssymb,amsmath}
\usepackage{proof,latexsym}
\usepackage[usenames]{color}
\usepackage{enumerate}
\usepackage{booktabs}
\usepackage{caption}
\captionsetup{justification=centering}
\usepackage{url}
\usepackage[para]{footmisc}
\usepackage{multirow}

\usepackage{tikz}
\usetikzlibrary{calc}
\usepackage{ifthen}
\usepackage{cite,balance}

\usepackage{subfigure}
\usepackage{wrapfig}
\usepackage{multirow}
\usepackage{rotating}
%\usepackage[leftno,noindent]{lgrind}
\usepackage{ifthen}
\usepackage{fullpage}
\usepackage{pifont}% http://ctan.org/pkg/pifont
\newcommand{\cmark}{\ding{51}}%
\newcommand{\xmark}{\ding{55}}%

\newcommand{\rot}[1]{\begin{rotate}{90} #1 \end{rotate}}
\newcommand{\rotb}[1]{\rot{\bf #1}}


\newboolean{showcomments}
\setboolean{showcomments}{true}
\ifthenelse{\boolean{showcomments}}
{ \newcommand{\mynote}[2]{
    \fbox{\bfseries\sffamily\scriptsize#1}
    {\small$\blacktriangleright$\textsf{\emph{#2}}$\blacktriangleleft$}}}
{ \newcommand{\mynote}[2]{}}


\def\cnf{\mathit{conf}}

\newenvironment{packed_enum}{
\begin{enumerate}
  \setlength{\itemsep}{1pt}
  \setlength{\parskip}{0pt}
  \setlength{\parsep}{0pt}
}{\end{enumerate}}

\newenvironment{packed_itemize}{
\begin{itemize}
  \setlength{\itemsep}{1pt}
  \setlength{\parskip}{0pt}
  \setlength{\parsep}{0pt}
}{\end{itemize}}

\def\concat{\mathit{+\!\!+}}


\newcommand{\todosimple}[2]
  {{\scriptsize \textbf{{#1} says: \color{red} {#2}}}}
\newcommand{\ssmnote}[1]{\todosimple{SM}{#1}}
\newcommand{\sdlnote}[1]{\todosimple{DL}{#1}}
\newcommand{\lx}[1]{\todosimple{LX}{#1}}
\newcommand{\tb}[1]{\todosimple{Bissyande}{#1}}


\newcommand{\task}[1]{\mynote{**Task**}{#1}}


\newcommand{\todo}[1]{\vspace{3mm}\begin{center}\fbox{
        \begin{minipage}{.7\linewidth} \small\begin{center}\fbox{To Do:}
        \end{center}#1\end{minipage}}\end{center}}

\newtheorem{theorem}{Theorem}
\newtheorem{definition}{Definition}

\pagestyle{plain}

\begin{document}

\pdfpagewidth 8.5in
\pdfpageheight 11.0in

\title{Optimizing the Mutant Execution Process for Efficient Mutation Analysis}

\author{\IEEEauthorblockN{Mike Papadakis, Tegawend\'e F. Bissyand\'e and Yves Le Traon}
    \IEEEauthorblockA{
    SnT, University of Luxembourg, Luxembourg\\
    michail.papadakis@uni.lu, tegawende.bissyande@uni.lu, yves.letraon@uni.lu\\
    }
}

\maketitle
\thispagestyle{plain}

\begin{abstract}
Mutation analysis is a powerful technique with several applications on
software development activities. Thus, mutants can be used to drive the
generation of test cases, the creation of oracles, the fault-localization
process and to assess the testing thoroughness. However, mutation analysis is
computationally expensive since it requires the test execution of a vast
number of mutants. This paper suggests the use of several optimizations
capable of reducing this overhead. The proposed optimizations have been
integrated into a mutation analysis tool for C and C++ programs. Empirical
results suggest that mutation analysis is feasible and practically applicable
to real word programs.

\end{abstract}



\section{Introduction}\label{sec.intro}
Software testing plays an important part in software development projects
where testing tasks can often account for over 50\% of the time expended and
of the development cost of a program~\cite{myers2011art}. Given the
investments on software testing, developers must ensure that the employed test
suites are adequate to thoroughly uncover hidden bugs in program code. In
practice, practitionners rely on (in)adequacy criteria, i.e., a set of test
obligations, to assess the adequacy/thoroughness of a test suite. Mutation
analysis is such a test adequacy criterion that is well-known by both
practitionners and researchers.



\section{Example}\label{sec.example}
\begin{table*}[ht!]
\scriptsize
\centering
\begin{tabular}{|l|c|c|c|c|c|c|c|c|}
\hline
& & & & & & &  & \\
& & & & & & & & \\
& & & & & & &  & \\
 \multirow{6}{*}{}& \multirow{6}{*}{\rotb{Mutant}} &  \multirow{6}{*}{\rotb{Statement}} & \multirow{6}{*}{\rotb{Execution 1}} &  \multirow{6}{*}{\rotb{Execution 2}} & \multirow{6}{*}{\rotb{Execution 3}} & \multirow{6}{*}{\rotb{Execution 4}} & \multirow{6}{*}{\rotb{Execution 5}} & \multirow{6}{*}{\rotb{Execution 6}}\\ 
& & & & & & & & \\
& & & & & & & & \\
& & & & & & &  & \\
 \hline
%\toprule
& & & & & & &  & \\
\multirow{3}{*}{$int\ mid(int\ x,\ int\ y,\ int\ z)\lbrace$} &\multirow{3}{*} {} & \multirow{3}{*}{} & \multirow{3}{*}{\rot{$3,3,5$}} & \multirow{3}{*}{\rot{$1,2,3$}} & \multirow{3}{*}{\rot{$3,2,1$}} & \multirow{3}{*}{\rot{$5,5,5$}} & \multirow{3}{*}{\rot{$5,3,4$}} & \multirow{3}{*}{\rot{$2,1,4$}} \\ %\midrule 
& & & & & & &  & \\
\hline
~~$int\ m;$ & & 1 &\xmark&\xmark&\xmark&\xmark&\xmark& \xmark \\ \hline
~~$m = z;$ & & 2 &\xmark&\xmark&\xmark&\xmark&\xmark& \xmark \\ \hline
~~$if(y < z)$ 			& 	& 3 &\xmark&\xmark&\xmark&\xmark&\xmark& \xmark\\ \hline
~~~~~$if(x < y)$ 		& 	& 4 &\xmark&\xmark&  	&  	&\xmark& \xmark \\ \hline
~~~~~~~~$m=y;$ 			& 	& 5 &   &\xmark&   &   &   &   \\ \hline
~~~~~$else\ if(x < z)$ 	& 	& 6 &\xmark&   &   &   &\xmark& \xmark \\ \hline
~~~~~~~~$m=x;$ 			& 	& 7 &\xmark&   &   &   &   & \xmark\\ \hline
~~$else$ 				& 	& 8 &   &   &\xmark&\xmark&   &   \\ \hline
~~~~~$if(x > y)$ 		& 	& 9 &   &   &\xmark&\xmark&   &   \\ \hline
~~~~~~~~$m=y;$ 			& 	& 10 &   &   &   &   &   &   \\ \hline
~~~~~$else\ if(x > z)$ 	& 	& 11 &   &   &\xmark&\xmark&   &   \\ \hline
~~~~~~~~$m=x;$ 			& 	& 12 &   &   &   &   &   &   \\ \hline
~~$return\ m;$ 			& 	& 13 &\xmark&\xmark&\xmark&\xmark&\xmark& \xmark\\ \hline
$\rbrace$ 				& 	& 	 &   &   &   &   &   &   \\ 
\hline
\hline
\hline
$int\ mid(int\ x,\ int\ y,\ int\ z)\lbrace$ & & & & & & & &\\ \hline
~~$int\ m;$ & & 1 & & & & & &   \\ \hline
~~$m = z;$ & & 2 & & & & & &   \\ \hline
~~\multirow{3}{*}{$if(y < z)$} & $M1. <\ \rightarrow\ <=$	& \multirow{3}{*}{3}& & & &\cmark& &  \\ \cline{2-2}  \cline{4-9}
 & $M2. <\ \rightarrow\ !=$	&  & & & & \cmark & & \\ \cline{2-2}  \cline{4-9}
 & $M3. <\ \rightarrow\ false$	&  & \cmark & \cmark & & & \cmark & \\ \hline
 ~~~~~\multirow{3}{*}{$if(x < y)$} & $M4. <\ \rightarrow\ <=$ & \multirow{3}{*}{4} &\cmark& &  & \cmark & &  \\ \cline{2-2}  \cline{4-9}
 & $M5. <\ \rightarrow\ !=$	&  & \cmark& &  	&  \cmark & & \\ \cline{2-2}  \cline{4-9}
 & $M6. <\ \rightarrow\ false$	&  & &\cmark&  	&  	& &   \\ \hline
~~~~~~~~$m=y;$ 			& 	& 5 &   & &   &   &   &   \\ \hline
~~~~~\multirow{3}{*}{$else\ if(x < z)$} 	& $M7. <\ \rightarrow\ <=$	& \multirow{3}{*}{6} & &   &   & \cmark  & & \\ \cline{2-2}  \cline{4-9}
& $M8. <\ \rightarrow\ !=$	& &  &   &   & \cmark  & &  \\ \cline{2-2}  \cline{4-9}
& $M9. <\ \rightarrow\ false$	&  &\cmark&  \cmark  &   &   & & \cmark \\ \hline
~~~~~~~~$m=x;$ & 	& 7 &  &   &   &   &   &  \\ \hline
~~$else$ 				& 	& 8 &   &   & & &   &   \\ \hline
~~~~~\multirow{3}{*}{$if(x > y)$}& $M10. >\ \rightarrow\ >=$	& \multirow{3}{*}{9} & \cmark  &   & &\cmark&   &   \\ \cline{2-2}  \cline{4-9}
& $M11. >\ \rightarrow\ !=$	&  & \cmark  &   & &\cmark&   &   \\ \cline{2-2}  \cline{4-9}
& $M12. >\ \rightarrow\ false$	&  &   &   &\cmark& & \cmark  &   \\ \hline
~~~~~~~~$m=y;$ 			& 	& 10 &   &   &   &   &   &   \\ \hline
~~~~~\multirow{3}{*}{$else\ if(x > z)$} & $M13. >\ \rightarrow\ >=$ & \multirow{3}{*}{11} &   &   & &\cmark&   &   \\ \cline{2-2}  \cline{4-9}
& $M14. >\ \rightarrow\ !=$ &  &  \cmark &   \cmark & &\cmark&   &  \cmark \\ \cline{2-2}  \cline{4-9}
& $M15. >\ \rightarrow\ false$ &  &   &   &\cmark& &  \cmark &   \\ \hline
~~~~~~~~$m=x;$ 			& 	& 12 &   &   &   &   &   &   \\ \hline
~~$return\ m;$ 			& 	& 13 & & & & & &  \\ \hline
$\rbrace$ 				& 	& 	 &   &   &   &   &   &   \\ 
\hline
%\bottomrule
\end{tabular}
%}
\caption{Illustration of mutation execution process optimization}
\end{table*}

\section{Approach}\label{sec.approach}
%\input{approach}

\section{Experimental Setup}\label{sec.experiment}
%\input{exp}

\section{Assessment}\label{sec.threats}
%\input{assessment}

\section{Related Work}\label{sec.related}
\input{related}

\section{Conclusion and Future Work} \label{sec.conclusion}
%\input{conclusion}



{
%\balance
%\scriptsize
%\def\baselinestretch{0.95}
\bibliographystyle{plain}
%\def\IEEEbibitemsep{0.92pt plus .5pt}
\bibliography{paper}
}

\end{document}
