% This is "sig-alternate.tex" V2.0 May 2012
% This file should be compiled with V2.5 of "sig-alternate.cls" May 2012
%
% This example file demonstrates the use of the 'sig-alternate.cls'
% V2.5 LaTeX2e document class file. It is for those submitting
% articles to ACM Conference Proceedings WHO DO NOT WISH TO
% STRICTLY ADHERE TO THE SIGS (PUBS-BOARD-ENDORSED) STYLE.
% The 'sig-alternate.cls' file will produce a similar-looking,
% albeit, 'tighter' paper resulting in, invariably, fewer pages.
%
% ----------------------------------------------------------------------------------------------------------------
% This .tex file (and associated .cls V2.5) produces:
%       1) The Permission Statement
%       2) The Conference (location) Info information
%       3) The Copyright Line with ACM data
%       4) NO page numbers
%
% as against the acm_proc_article-sp.cls file which
% DOES NOT produce 1) thru' 3) above.
%
% Using 'sig-alternate.cls' you have control, however, from within
% the source .tex file, over both the CopyrightYear
% (defaulted to 200X) and the ACM Copyright Data
% (defaulted to X-XXXXX-XX-X/XX/XX).
% e.g.
% \CopyrightYear{2007} will cause 2007 to appear in the copyright line.
% \crdata{0-12345-67-8/90/12} will cause 0-12345-67-8/90/12 to appear in the copyright line.
%
% ---------------------------------------------------------------------------------------------------------------
% This .tex source is an example which *does* use
% the .bib file (from which the .bbl file % is produced).
% REMEMBER HOWEVER: After having produced the .bbl file,
% and prior to final submission, you *NEED* to 'insert'
% your .bbl file into your source .tex file so as to provide
% ONE 'self-contained' source file.
%
% ================= IF YOU HAVE QUESTIONS =======================
% Questions regarding the SIGS styles, SIGS policies and
% procedures, Conferences etc. should be sent to
% Adrienne Griscti (griscti@acm.org)
%
% Technical questions _only_ to
% Gerald Murray (murray@hq.acm.org)
% ===============================================================
%
% For tracking purposes - this is V2.0 - May 2012

\documentclass{sig-alternate}
%\usepackage{graphicx}
%\usepackage{epstopdf}

%\usepackage{amssymb,amsmath}
%\usepackage{proof,latexsym}
%\usepackage[usenames]{color}
%\usepackage{enumerate}
%\usepackage{booktabs}
%\usepackage{caption}
%\captionsetup{justification=centering}
%\usepackage{url}
%\usepackage[para]{footmisc}
\usepackage{multirow}

%\usepackage{tikz}
%\usetikzlibrary{calc}
%\usepackage{ifthen}
%\usepackage{cite,balance}

%\usepackage{subfigure}
%\usepackage{wrapfig}

\usepackage{rotating}
%\usepackage[leftno,noindent]{lgrind}
%\usepackage{ifthen}
%\usepackage{fullpage}
\usepackage{pifont}% http://ctan.org/pkg/pifont
\newcommand{\cmark}{\ding{51}}%
\newcommand{\xmark}{\ding{55}}%

\newcommand{\rot}[1]{\begin{rotate}{90} #1 \end{rotate}}
\newcommand{\rotb}[1]{\rot{\bf #1}}


\newboolean{showcomments}
\setboolean{showcomments}{true}
\ifthenelse{\boolean{showcomments}}
{ \newcommand{\mynote}[2]{
    \fbox{\bfseries\sffamily\scriptsize#1}
    {\small$\blacktriangleright$\textsf{\emph{#2}}$\blacktriangleleft$}}}
{ \newcommand{\mynote}[2]{}}


%\def\cnf{\mathit{conf}}

\newenvironment{packed_enum}{
\begin{enumerate}
  \setlength{\itemsep}{1pt}
  \setlength{\parskip}{0pt}
  \setlength{\parsep}{0pt}
}{\end{enumerate}}

\newenvironment{packed_itemize}{
\begin{itemize}
  \setlength{\itemsep}{1pt}
  \setlength{\parskip}{0pt}
  \setlength{\parsep}{0pt}
}{\end{itemize}}

\def\concat{\mathit{+\!\!+}}


\newcommand{\todosimple}[2]
  {{\scriptsize \textbf{{#1} says: \color{red} {#2}}}}
\newcommand{\ssmnote}[1]{\todosimple{SM}{#1}}
\newcommand{\sdlnote}[1]{\todosimple{DL}{#1}}
\newcommand{\lx}[1]{\todosimple{LX}{#1}}
\newcommand{\tb}[1]{\todosimple{Bissyande}{#1}}


\newcommand{\task}[1]{\mynote{**Task**}{#1}}


\newcommand{\todo}[1]{\vspace{3mm}\begin{center}\fbox{
        \begin{minipage}{.7\linewidth} \small\begin{center}\fbox{To Do:}
        \end{center}#1\end{minipage}}\end{center}}

\newtheorem{theorem}{Theorem}
\newtheorem{definition}{Definition}

\pagestyle{plain}

\begin{document}
%
% --- Author Metadata here ---
\conferenceinfo{ISSTA}{'14 San Jose, California USA}
%\CopyrightYear{2007} % Allows default copyright year (20XX) to be over-ridden - IF NEED BE.
%\crdata{0-12345-67-8/90/01}  % Allows default copyright data (0-89791-88-6/97/05) to be over-ridden - IF NEED BE.
% --- End of Author Metadata ---

\title{Optimizing the Mutant Execution Process for Efficient Mutation Analysis}
%\subtitle{[Extended Abstract]
%\titlenote{A full version of this paper is available as
%\textit{Author's Guide to Preparing ACM SIG Proceedings Using
%\LaTeX$2_\epsilon$\ and BibTeX} at
%\texttt{www.acm.org/eaddress.htm}}}
%
% You need the command \numberofauthors to handle the 'placement
% and alignment' of the authors beneath the title.
%
% For aesthetic reasons, we recommend 'three authors at a time'
% i.e. three 'name/affiliation blocks' be placed beneath the title.
%
% NOTE: You are NOT restricted in how many 'rows' of
% "name/affiliations" may appear. We just ask that you restrict
% the number of 'columns' to three.
%
% Because of the available 'opening page real-estate'
% we ask you to refrain from putting more than six authors
% (two rows with three columns) beneath the article title.
% More than six makes the first-page appear very cluttered indeed.
%
% Use the \alignauthor commands to handle the names
% and affiliations for an 'aesthetic maximum' of six authors.
% Add names, affiliations, addresses for
% the seventh etc. author(s) as the argument for the
% \additionalauthors command.
% These 'additional authors' will be output/set for you
% without further effort on your part as the last section in
% the body of your article BEFORE References or any Appendices.


\numberofauthors{6} %  in this sample file, there are a *total*
% of EIGHT authors. SIX appear on the 'first-page' (for formatting
% reasons) and the remaining two appear in the \additionalauthors section.
%
\author{
% You can go ahead and credit any number of authors here,
% e.g. one 'row of three' or two rows (consisting of one row of three
% and a second row of one, two or three).
%
% The command \alignauthor (no curly braces needed) should
% precede each author name, affiliation/snail-mail address and
% e-mail address. Additionally, tag each line of
% affiliation/address with \affaddr, and tag the
% e-mail address with \email.
%
% 1st. author
\alignauthor
Ben Trovato\titlenote{Dr.~Trovato insisted his name be first.}\\
       \affaddr{Institute for Clarity in Documentation}\\
       \affaddr{1932 Wallamaloo Lane}\\
       \affaddr{Wallamaloo, New Zealand}\\
       \email{trovato@corporation.com}
% 2nd. author
\alignauthor
G.K.M. Tobin\titlenote{The secretary disavows
any knowledge of this author's actions.}\\
       \affaddr{Institute for Clarity in Documentation}\\
       \affaddr{P.O. Box 1212}\\
       \affaddr{Dublin, Ohio 43017-6221}\\
       \email{webmaster@marysville-ohio.com}
% 3rd. author
\alignauthor Lars Th{\o}rv{\"a}ld\titlenote{This author is the
one who did all the really hard work.}\\
       \affaddr{The Th{\o}rv{\"a}ld Group}\\
       \affaddr{1 Th{\o}rv{\"a}ld Circle}\\
       \affaddr{Hekla, Iceland}\\
       \email{larst@affiliation.org}
\and  % use '\and' if you need 'another row' of author names
% 4th. author
\alignauthor Lawrence P. Leipuner\\
       \affaddr{Brookhaven Laboratories}\\
       \affaddr{Brookhaven National Lab}\\
       \affaddr{P.O. Box 5000}\\
       \email{lleipuner@researchlabs.org}
% 5th. author
\alignauthor Sean Fogarty\\
       \affaddr{NASA Ames Research Center}\\
       \affaddr{Moffett Field}\\
       \affaddr{California 94035}\\
       \email{fogartys@amesres.org}
% 6th. author
\alignauthor Charles Palmer\\
       \affaddr{Palmer Research Laboratories}\\
       \affaddr{8600 Datapoint Drive}\\
       \affaddr{San Antonio, Texas 78229}\\
       \email{cpalmer@prl.com}
}
% There's nothing stopping you putting the seventh, eighth, etc.
% author on the opening page (as the 'third row') but we ask,
% for aesthetic reasons that you place these 'additional authors'
% in the \additional authors block, viz.




% There's nothing stopping you putting the seventh, eighth, etc.
% author on the opening page (as the 'third row') but we ask,
% for aesthetic reasons that you place these 'additional authors'
% in the \additional authors block, viz.
%\additionalauthors{Additional authors: John Smith (The Th{\o}rv{\"a}ld Group,
%email: {\texttt{jsmith@affiliation.org}}) and Julius P.~Kumquat
%(The Kumquat Consortium, email: {\texttt{jpkumquat@consortium.net}}).}
%\date{30 July 1999}
% Just remember to make sure that the TOTAL number of authors
% is the number that will appear on the first page PLUS the
% number that will appear in the \additionalauthors section.

\maketitle
\thispagestyle{plain}

\begin{abstract}
Mutation analysis is a powerful technique with several applications on
software development activities. Thus, mutants can be used to drive the
generation of test cases, the creation of oracles, the fault-localization
process and to assess the testing thoroughness. However, mutation analysis is
computationally expensive since it requires the test execution of a vast
number of mutants. This paper suggests the use of several optimizations
capable of reducing this overhead. The proposed optimizations have been
integrated into a mutation analysis tool for C and C++ programs. Empirical
results suggest that mutation analysis is feasible and practically applicable
to real word programs.

\end{abstract}



\section{Introduction}\label{sec.intro}
Software testing plays an important part in software development projects
where testing tasks can often account for over 50\% of the time expended and
of the development cost of a program~\cite{myers2011art}. Given the
investments on software testing, developers must ensure that the employed test
suites are adequate to thoroughly uncover hidden bugs in program code. In
practice, practitionners rely on (in)adequacy criteria, i.e., a set of test
obligations, to assess the adequacy/thoroughness of a test suite. Mutation
analysis is such a test adequacy criterion that is well-known by both
practitionners and researchers.



\section{Example}\label{sec.example}
\begin{table*}[ht!]
\scriptsize
\centering
\begin{tabular}{|l|c|c|c|c|c|c|c|c|}
\hline
& & & & & & &  & \\
& & & & & & & & \\
& & & & & & &  & \\
 \multirow{6}{*}{}& \multirow{6}{*}{\rotb{Mutant}} &  \multirow{6}{*}{\rotb{Statement}} & \multirow{6}{*}{\rotb{Execution 1}} &  \multirow{6}{*}{\rotb{Execution 2}} & \multirow{6}{*}{\rotb{Execution 3}} & \multirow{6}{*}{\rotb{Execution 4}} & \multirow{6}{*}{\rotb{Execution 5}} & \multirow{6}{*}{\rotb{Execution 6}}\\ 
& & & & & & & & \\
& & & & & & & & \\
& & & & & & &  & \\
 \hline
%\toprule
& & & & & & &  & \\
\multirow{3}{*}{$int\ mid(int\ x,\ int\ y,\ int\ z)\lbrace$} &\multirow{3}{*} {} & \multirow{3}{*}{} & \multirow{3}{*}{\rot{$3,3,5$}} & \multirow{3}{*}{\rot{$1,2,3$}} & \multirow{3}{*}{\rot{$3,2,1$}} & \multirow{3}{*}{\rot{$5,5,5$}} & \multirow{3}{*}{\rot{$5,3,4$}} & \multirow{3}{*}{\rot{$2,1,4$}} \\ %\midrule 
& & & & & & &  & \\
\hline
~~$int\ m;$ & & 1 &\xmark&\xmark&\xmark&\xmark&\xmark& \xmark \\ \hline
~~$m = z;$ & & 2 &\xmark&\xmark&\xmark&\xmark&\xmark& \xmark \\ \hline
~~$if(y < z)$ 			& 	& 3 &\xmark&\xmark&\xmark&\xmark&\xmark& \xmark\\ \hline
~~~~~$if(x < y)$ 		& 	& 4 &\xmark&\xmark&  	&  	&\xmark& \xmark \\ \hline
~~~~~~~~$m=y;$ 			& 	& 5 &   &\xmark&   &   &   &   \\ \hline
~~~~~$else\ if(x < z)$ 	& 	& 6 &\xmark&   &   &   &\xmark& \xmark \\ \hline
~~~~~~~~$m=x;$ 			& 	& 7 &\xmark&   &   &   &   & \xmark\\ \hline
~~$else$ 				& 	& 8 &   &   &\xmark&\xmark&   &   \\ \hline
~~~~~$if(x > y)$ 		& 	& 9 &   &   &\xmark&\xmark&   &   \\ \hline
~~~~~~~~$m=y;$ 			& 	& 10 &   &   &   &   &   &   \\ \hline
~~~~~$else\ if(x > z)$ 	& 	& 11 &   &   &\xmark&\xmark&   &   \\ \hline
~~~~~~~~$m=x;$ 			& 	& 12 &   &   &   &   &   &   \\ \hline
~~$return\ m;$ 			& 	& 13 &\xmark&\xmark&\xmark&\xmark&\xmark& \xmark\\ \hline
$\rbrace$ 				& 	& 	 &   &   &   &   &   &   \\ 
\hline
\hline
\hline
$int\ mid(int\ x,\ int\ y,\ int\ z)\lbrace$ & & & & & & & &\\ \hline
~~$int\ m;$ & & 1 & & & & & &   \\ \hline
~~$m = z;$ & & 2 & & & & & &   \\ \hline
~~\multirow{3}{*}{$if(y < z)$} & $M1. <\ \rightarrow\ <=$	& \multirow{3}{*}{3}& & & &\cmark& &  \\ \cline{2-2}  \cline{4-9}
 & $M2. <\ \rightarrow\ !=$	&  & & & & \cmark & & \\ \cline{2-2}  \cline{4-9}
 & $M3. <\ \rightarrow\ false$	&  & \cmark & \cmark & & & \cmark & \\ \hline
 ~~~~~\multirow{3}{*}{$if(x < y)$} & $M4. <\ \rightarrow\ <=$ & \multirow{3}{*}{4} &\cmark& &  & \cmark & &  \\ \cline{2-2}  \cline{4-9}
 & $M5. <\ \rightarrow\ !=$	&  & \cmark& &  	&  \cmark & & \\ \cline{2-2}  \cline{4-9}
 & $M6. <\ \rightarrow\ false$	&  & &\cmark&  	&  	& &   \\ \hline
~~~~~~~~$m=y;$ 			& 	& 5 &   & &   &   &   &   \\ \hline
~~~~~\multirow{3}{*}{$else\ if(x < z)$} 	& $M7. <\ \rightarrow\ <=$	& \multirow{3}{*}{6} & &   &   & \cmark  & & \\ \cline{2-2}  \cline{4-9}
& $M8. <\ \rightarrow\ !=$	& &  &   &   & \cmark  & &  \\ \cline{2-2}  \cline{4-9}
& $M9. <\ \rightarrow\ false$	&  &\cmark&  \cmark  &   &   & & \cmark \\ \hline
~~~~~~~~$m=x;$ & 	& 7 &  &   &   &   &   &  \\ \hline
~~$else$ 				& 	& 8 &   &   & & &   &   \\ \hline
~~~~~\multirow{3}{*}{$if(x > y)$}& $M10. >\ \rightarrow\ >=$	& \multirow{3}{*}{9} & \cmark  &   & &\cmark&   &   \\ \cline{2-2}  \cline{4-9}
& $M11. >\ \rightarrow\ !=$	&  & \cmark  &   & &\cmark&   &   \\ \cline{2-2}  \cline{4-9}
& $M12. >\ \rightarrow\ false$	&  &   &   &\cmark& & \cmark  &   \\ \hline
~~~~~~~~$m=y;$ 			& 	& 10 &   &   &   &   &   &   \\ \hline
~~~~~\multirow{3}{*}{$else\ if(x > z)$} & $M13. >\ \rightarrow\ >=$ & \multirow{3}{*}{11} &   &   & &\cmark&   &   \\ \cline{2-2}  \cline{4-9}
& $M14. >\ \rightarrow\ !=$ &  &  \cmark &   \cmark & &\cmark&   &  \cmark \\ \cline{2-2}  \cline{4-9}
& $M15. >\ \rightarrow\ false$ &  &   &   &\cmark& &  \cmark &   \\ \hline
~~~~~~~~$m=x;$ 			& 	& 12 &   &   &   &   &   &   \\ \hline
~~$return\ m;$ 			& 	& 13 & & & & & &  \\ \hline
$\rbrace$ 				& 	& 	 &   &   &   &   &   &   \\ 
\hline
%\bottomrule
\end{tabular}
%}
\caption{Illustration of mutation execution process optimization}
\end{table*}

\section{Approach}\label{sec.approach}
%\input{approach}

\section{Experimental Setup}\label{sec.experiment}
%\input{exp}

\section{Assessment}\label{sec.threats}
%\input{assessment}

\section{Related Work}\label{sec.related}
\input{related}

\section{Conclusion and Future Work} \label{sec.conclusion}
%\input{conclusion}


{
%\balance
%\scriptsize
%\def\baselinestretch{0.95}
\bibliographystyle{plain}
%\def\IEEEbibitemsep{0.92pt plus .5pt}
\bibliography{paper}
}

\end{document}
